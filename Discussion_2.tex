\documentclass[12pt]{article}
\usepackage{xeCJK}%preamble part
\usepackage{graphicx}
\usepackage{indentfirst}
\usepackage[a4paper, inner=1.5cm, outer=3cm, top=2cm, bottom=3cm, bindingoffset=1cm]{geometry}
\usepackage{epstopdf}
\usepackage{listings}
\usepackage{array}
\usepackage{fontspec}
\usepackage{gensymb}
\usepackage{amsmath}
\usepackage[citecolor=blue]{hyperref}

\usepackage{makecell}
\usepackage[lofdepth,lotdepth]{subfig}
\setCJKmainfont[BoldFont={SimHei}]{SimSun}
\setCJKmonofont{SimSun}
\setmainfont{Times New Roman}
\newCJKfontfamily[hei]\heiti{SimHei}
\setlength{\extrarowheight}{4pt}
\setlength{\parindent}{1cm}
\begin{document}
\title{\textbf{\fontsize{15.75pt}{\baselineskip}{和沈老师讨论的进一步结果}}} 

\author{\fontsize{12pt}{\baselineskip}{数33 赵丰}}
\maketitle
\large
\section{\textbf{\fontsize{12pt}{\baselineskip}{仿真结果}}}
采用前面所述的方法进行仿真,只考虑单位点的情形,每个位点采四个数据,01序列由之前所述的Markov 链生成,
将贝叶斯方法与传统的t检验进行对比,根据仿真实验的参数,贝叶斯方法最小错误率为$2P(Z<-1)\approx 0.32$
对长为1000的链,由于其出现01的概率相等,所以用贝叶斯方法的错误率实际上只有理论值的一半,为0.159左右。
传统的t检验由于只是控制第一类错误的概率,没有考虑到$\theta \ne 0$时$\bar{x}$具体的分布,因此没有充分利用
$\theta$的先验的非0即1的信息,在样本值为1000的情况下,通过调整显著性水平$\alpha$的值,总的错误率只能降低到
27\%左右。相比较而言,贝叶斯统计更有优势。
\section{\textbf{\fontsize{12pt}{\baselineskip}{参数估计}}}
虽然01序列的各个位点有很强的相关性,但一旦01序列给定,不同位点的观测值彼此独立,如前节假定
$X \sim N(\mu_1,\sigma_1^2) | \theta=0),X \sim N(\mu_2,\sigma_2^2) | \theta=1)$.
通过对RNA序列的统计,可以得到单双链的比例p,即$\theta \sim B(p)$。
由于RNA序列一般较长,即使单个位点采集的样本点很少,但总共的信息很多,我们可以利用全局的信息对
$\mu_1,\mu_2,\sigma_1,\sigma_2$用矩估计的方法做出估计。
方法如下:
将所有位点的观测值$X_1,X_2,..X_n$分别求1到4阶矩,$X_i$之间彼此独立,由于n很大,由大数定律可得:
\begin{equation}
\frac{\sum_{i=1}^n X^j_i}{n} \approx (1-p)E(X^j|\theta=0)+pE(X^j|\theta=1),j=1,2,3,4
\end{equation}
由正态分布的密度函数可分别算出其前4阶原点矩,由此得如下关于$\mu_1,\mu_2,\sigma_1,\sigma_2$
的4元方程:
\begin{equation}
\begin{split}
&\frac{\sum_{i=1}^n X_i}{n} \approx (1-p)\mu_1+p\mu_2 \\
&\frac{\sum_{i=2}^n X^2_i}{n} \approx (1-p)(\mu_1^2+\sigma_1^2)+p(\mu_2^2+\sigma_2^2) \\
&\frac{\sum_{i=2}^n X^3_i}{n} \approx (1-p)(\mu_1^3+3\sigma_1^2\mu_1)+p(\mu_2^3+3\sigma_2^2\mu_2) \\
&\frac{\sum_{i=2}^n X^4_i}{n} \approx (1-p)(\mu_1^4+6\sigma_1^2\mu_1^2+3\sigma_1^4)+p(\mu_2^4+6\sigma_2^2\mu_2^2+3\sigma_2^4)
\end{split}
\end{equation}
根据原始数据可以求出上述方程组左边的值,由此解出待估计参数。
\section{\textbf{\fontsize{12pt}{\baselineskip}{参数估计数值实验}}}
使用李盼提供的数据(只用RT)读取某实验条件下实验组2组,对照组两组,每组数据长度均为1870。
先将对照组数据取平均值,再用实验组除以取平均值后的对照组,R代码如下:
\begin{lstlisting}{language=R}
x1<-scan('cy_D1.rt',what=numeric(0),n=1e6)
x2<-scan('cy_D2.rt',what=numeric(0),n=1e6)
x3<-scan('cy_N1.rt',what=numeric(0),n=1e6)
x4<-scan('cy_N2.rt',what=numeric(0),n=1e6)
x_case_1<-2*x3/(x1+x2);
x_case_2<-2*x4/(x1+x2);
#cor(x1,x2)=0.93
#cor(x3,x4)=0.99
\end{lstlisting}
进一步计算两组case的统计信息得下表:
\begin{table}[!ht]
\begin{tabular}{ccc}
\hline
参数 & x\_case\_1 & x\_case\_2 \\
\hline
最小值 & 0.083 & 0.108 \\
最大值 & 5.882 & 5.457 \\
一阶原点矩 &1.729 & 1.807 \\
二阶原点矩 &4.106 & 4.377\\
三阶原点矩 &11.963 &12.681\\
四阶原点矩 &40.513 &41.610\\
\hline
\end{tabular}
\end{table}
从上表可以看出两组数据相差不大,可以用将两组数据合并用来估计四个参数。有icshape实验值按0.5的阈值二值化得01序列,计算出单链比例为0.53,近似取p=0.5.代入已知数据,求解上述非线性方程组得:
\[\mu_1=1.165,\mu_2=2.371,\sigma_1^2=0.269,\sigma_2^2=1.236\]
下面利用单位点的两组数据根据前面的贝叶斯统计模型做判决,R代码如下:
\begin{lstlisting}{language=R}
mu_1=2.3711
mu_2=1.1649
sigma_1=1.2362
sigma_2=0.2686
n=2
my_y=c()
for(i in 1:length(x_case_1)){
xbar=(x_case_1[i]+x_case_2[i])/2;
p1=log(sigma_1)+n*(xbar-mu_1)^2/sigma_1^2;
p2=log(sigma_2)+n*(xbar-mu_2)^2/sigma_2^2;
if(p1>p2){
my_y=c(my_y,1);
}
else{my_y=c(my_y,0);}#0 is single-chain
}
\end{lstlisting}
\section{\textbf{\fontsize{12pt}{\baselineskip}{结果比较}}}
使用上述贝叶斯统计的方法获得my\_y01序列,其与标准结构的差别率为51.9\%,而使用icshape的方法获得y序列,
其与标准结构的差别率为36\%.比较Bayes方法和icshape的方法结果,差别率为51\%.
但是比较x\_i\_case和结果的相关性,icshape的方法只有-0.19(RT值越大,更倾向$\theta=0$,即y[i]=0所以是负相关)
而Bayes method 达到-0.34,可见Bayes method 更优。主要问题是4组数据变2组用直接相除法可能不太理想。如果在原来实验组-$\alpha$乘以对照组的基础上用Bayes method效果可能比较好。

如果尝试用$\log(case)=A*\log(control)+B+signal$对模型进行拟合:
则R代码如下:
\begin{lstlisting}{language=R}
control<-(x1+x2)/2;
plot(x=log(control),y=log(x3),pch=20,col='red',
ylab='log(case)')
abline(lm(log(x3)~log(control)),col='blue')
my_coff=lm(log(x3)~log(control))
intercept=my_coff$coefficients[1]
slope=my_coff$coefficients[2]
signal_value_1=log(x3)-(intercept+slope*log(control))
signal_value_2=log(x3)-(intercept+slope*log(control))
\end{lstlisting}
上面做出的$\log(case) \sim \log(control)$如下所示
\begin{figure}[!ht]
\centering
\caption{线性回归拟合}
\includegraphics[width=400pt]{Plotting3.eps}
\end{figure}

\newpage
对$signal_value_1$做直方图如下:
\begin{figure}[!ht]
\centering
\caption{$signal_value_1$的直方图}
\includegraphics[width=400pt]{Plotting4.eps}
\end{figure}


由上图可以看出,$signal_value_1$有两个峰,可以近似认为是两组正态样本混叠在一起。
使用上述得到signal的方法用贝叶斯统计估计原01序列,得my\_y,icshape给出的01序列与signal\_value\_1的相关性只有-0.23,而用用贝叶斯统计的方法结果与signal\_value\_1的相关性可达-0.45.两种方法差别率为43\%。




\section{\textbf{\fontsize{12pt}{\baselineskip}{参考文献}}}
\begin{thebibliography}{}
\bibitem{Bib1}    
 生物信息学
 
\end{thebibliography}
\end{document}