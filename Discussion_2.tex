\documentclass[12pt]{article}
\usepackage{xeCJK}%preamble part
\usepackage{graphicx}
\usepackage{indentfirst}
\usepackage[a4paper, inner=1.5cm, outer=3cm, top=2cm, bottom=3cm, bindingoffset=1cm]{geometry}
\usepackage{epstopdf}
\usepackage{listings}
\usepackage{array}
\usepackage{fontspec}
\usepackage{gensymb}
\usepackage{amsmath}
\usepackage[citecolor=blue]{hyperref}

\usepackage{makecell}
\usepackage[lofdepth,lotdepth]{subfig}
\setCJKmainfont[BoldFont={SimHei}]{SimSun}
\setCJKmonofont{SimSun}
\setmainfont{Times New Roman}
\newCJKfontfamily[hei]\heiti{SimHei}
\setlength{\extrarowheight}{4pt}
\setlength{\parindent}{1cm}
\begin{document}
\title{\textbf{\fontsize{15.75pt}{\baselineskip}{和沈老师讨论的进一步结果}}} 

\author{\fontsize{12pt}{\baselineskip}{数33 赵丰}}
\maketitle
\large
\section{\textbf{\fontsize{12pt}{\baselineskip}{仿真结果}}}
采用前面所述的方法进行仿真,只考虑单位点的情形,每个位点采四个数据,01序列由之前所述的Markov 链生成,
将贝叶斯方法与传统的t检验进行对比,根据仿真实验的参数,贝叶斯方法最小错误率为$2P(Z<-1)\approx 0.32$
对长为1000的链,由于其出现01的概率相等,所以用贝叶斯方法的错误率实际上只有理论值的一半,为0.159左右。
传统的t检验由于只是控制第一类错误的概率,没有考虑到$\theta \ne 0$时$\bar{x}$具体的分布,因此没有充分利用
$\theta$的先验的非0即1的信息,在样本值为1000的情况下,通过调整显著性水平$\alpha$的值,总的错误率只能降低到
27\%左右。相比较而言,贝叶斯统计更有优势。
\section{\textbf{\fontsize{12pt}{\baselineskip}{参数估计}}}
虽然01序列的各个位点有很强的相关性,但一旦01序列给定,不同位点的观测值彼此独立,如前节假定
$X \sim N(\mu_1,\sigma_1^2) | \theta=0),X \sim N(\mu_2,\sigma_2^2) | \theta=1)$.
通过对RNA序列的统计,可以得到单双链的比例p,即$\theta \sim B(p)$。
由于RNA序列一般较长,即使单个位点采集的样本点很少,但总共的信息很多,我们可以利用全局的信息对
$\mu_1,\mu_2,\sigma_1,\sigma_2$用矩估计的方法做出估计。
方法如下:
将所有位点的观测值$X_1,X_2,..X_n$分别求1到4阶矩,$X_i$之间彼此独立,由于n很大,由大数定律可得:
\begin{equation}
\frac{\sum_{i=1}^n X^j_i}{n} \approx (1-p)E(X^j|\theta=0)+pE(X^j|\theta=1),j=1,2,3,4
\end{equation}
由正态分布的密度函数可分别算出其前4阶原点矩,由此得如下关于$\mu_1,\mu_2,\sigma_1,\sigma_2$
的4元方程:
\begin{equation}
\begin{split}
&\frac{\sum_{i=1}^n X_i}{n} \approx (1-p)\mu_1+p\mu_2 \\
&\frac{\sum_{i=2}^n X^2_i}{n} \approx (1-p)(\mu_1^2+\sigma_1^2)+p(\mu_2^2+\sigma_2^2) \\
&\frac{\sum_{i=2}^n X^3_i}{n} \approx (1-p)(\mu_1^3+3\sigma_1^2\mu_1)+p(\mu_2^3+3\sigma_2^2\mu_2) \\
&\frac{\sum_{i=2}^n X^4_i}{n} \approx (1-p)(\mu_1^4+6\sigma_1^2\mu_1^2+3\sigma_1^4)+p(\mu_2^4+6\sigma_2^2\mu_2^2+3\sigma_2^4)
\end{split}
\end{equation}
根据原始数据可以求出上述方程组左边的值,由此解出待估计参数。
\section{\textbf{\fontsize{12pt}{\baselineskip}{参数估计数值实验}}}
使用李盼提供的数据(只用RT)读取某实验条件下实验组2组,对照组两组,每组数据长度均为1870。
先将对照组数据取平均值,再用实验组除以取平均值后的对照组,R代码如下:
\begin{lstlisting}{language=R}
x1<-scan('cy_D1.rt',what=numeric(0),n=1e6)
x2<-scan('cy_D2.rt',what=numeric(0),n=1e6)
x3<-scan('cy_N1.rt',what=numeric(0),n=1e6)
x4<-scan('cy_N2.rt',what=numeric(0),n=1e6)
x_case_1<-2*x3/(x1+x2);
x_case_2<-2*x4/(x1+x2);
#cor(x1,x2)=0.93
#cor(x3,x4)=0.99
\end{lstlisting}
进一步计算两组case的统计信息得下表:
\begin{table}[!ht]
\begin{tabular}{ccc}
\hline
参数 & x\_case\_1 & x\_case\_2 \\
\hline
最小值 & 0.083 & 0.108 \\
最大值 & 5.882 & 5.457 \\
一阶原点矩 &1.729 & 1.807 \\
二阶原点矩 &4.106 & 4.377\\
三阶原点矩 &11.963 &12.681\\
四阶原点矩 &40.513 &41.610\\
\hline
\end{tabular}
\end{table}
从上表可以看出两组数据相差不大,可以用将两组数据合并用来估计四个参数。有icshape实验值按0.5的阈值二值化得01序列,计算出单链比例为0.53,近似取p=0.5.代入已知数据,求解上述非线性方程组得:
\[\mu_1=1.53,\mu_2=2.01,\sigma_1^2=-0.167,\sigma_2^2=2.28\]
$\sigma_1^2$解出现负数,不合理,如果不修正模型,只能放宽解的条件,将方程转化为带约束的优化问题。
\section{\textbf{\fontsize{12pt}{\baselineskip}{Appendix 1:拒绝域表达式的推导}}}
\section{\textbf{\fontsize{12pt}{\baselineskip}{参考文献}}}
\begin{thebibliography}{}
\bibitem{Bib1}    
 生物信息学
 
\end{thebibliography}
\end{document}