\documentclass[12pt,twocolumn]{article}
\usepackage{xeCJK}%preamble part
\usepackage{graphicx}
\usepackage{indentfirst}
\usepackage[a4paper, inner=1.5cm, outer=3cm, top=2cm, bottom=3cm, bindingoffset=1cm]{geometry}
\usepackage{epstopdf}
\usepackage{array}
\usepackage{fontspec}
\usepackage{gensymb}
\usepackage[citecolor=blue]{hyperref}
\usepackage{amsmath}
\usepackage{makecell}
\usepackage[lofdepth,lotdepth]{subfig}
\setCJKmainfont[BoldFont={SimHei}]{SimSun}
\setCJKmonofont{SimSun}
\setmainfont{Times New Roman}
\newCJKfontfamily[hei]\heiti{SimHei}
\setlength{\extrarowheight}{4pt}
\setlength{\parindent}{1cm}
\begin{document}
\large
\twocolumn
\setlength{\columnseprule}{1pt}
均值和方差是很容易理解的概念,实验多次测量,测量值的均值作为估计,样本方差的大小衡量数据的可靠性,我觉得对于简单的统计问题(单样本,抽样之间独立),搬出理论框架并不会比凭感觉乱搞多增加多少可靠性;对于比较复杂的问题(数据维数多,相关性错综复杂),感觉帮不上太多的忙,主要要靠理论指导,然而我对其理论框架并不甚了解,故以下的讨论只好举出非常简单的例子,讨论推断性统计学的框架均值和方差的问题,大概至少要分三种情况:

\begin{enumerate}
\item 区间估计
\item 点估计
\item 假设检验
\end{enumerate}

先讨论第一种情形,大学物理绪论课(参见\cite{Bib1})讲由于随机误差的存在,结果要写成$\bar{x}+t_{\xi}(\nu)S_{\bar{x}}$的形式,就是区间估计的表达方式。
意思是说,假设待测物理量是一个随机变量X,X服从正态分布$N(\mu,\sigma)$。现在要测它的均值E(X)((就是估计参数$\mu$))。先给定一个置信度$\alpha$(比如95\%),那么我有$\alpha$的把握断言E(X)落在
区间$[\bar{X}-t_{\xi}(\nu)S_{\bar{X}},\bar{X}+t_{\xi}(\nu)S_{\bar{X}}]$,关于$\alpha$,实验重复次数和区间大小的关系具体表达式就不给出了,只是定性地说,$alpha$越接近1,$t_{\xi}(\nu)$会越大,要估计的值落在一个更大的区间里,极端情况是说我有100\%的把握断言E(X)落在
区间$[\bar{X}-\infty,\bar{X}+\infty]$,这样Inference也没什么意义了;实验重复次数越多$S_{\bar{X}}$会越小,要估计的值落在一个更小的区间里,极端情况是说只要实验重复足够多次,我就有$\alpha$的把握断言E(X)一定是$\bar{X}$。

上面举了X是连续性随机变量的例子,对于离散的情形,一般会考虑Bernoulli分布B(p)。同样是先给定一个置信度$\alpha$,
那么我有$\alpha$的把握断言E(X)(就是估计参数p)落在某个区间,关于这个区间构造最简单的方法是:
\begin{equation*}
[\bar{x}-z\sqrt{\frac{\bar{x}(1-\bar{x})}{n}},\bar{x}+z\sqrt{\frac{\bar{x}(1-\bar{x})}{n}}] \quad^{\cite{Bib2}}
\end{equation*}
这个估计并不好,一方面只有当实验重复次数足够多时近似程度才比较好,另一方面当p接近0或者1时区间长度会很窄,估计效果很差$^{\cite{Bib3}}$。
\onecolumn

再讨论第一种情形,即只有两种判决结果的假设检验问题,此处仍然分正态性样本和Bernoulli实验样本两种最常用的情况分别讨论。
对于正态性样本$X$,N次实验的结果得到$X_i,i=1,2,...N$,为$i.i.d$的$r.v.$,为检验样本均值是否为$\mu_0$,构造统计量
\begin{equation}
T=\frac{\bar{x}-\mu_0}{\sqrt{S^2_X/N}}
\end{equation}
$H_0:\mu=\mu_0 versus H_1:\mu \not \mu_0$

\begin{thebibliography}{}
\bibitem{Bib1}大学物理实验误差与数据处理章节
\bibitem{Bib2} \url{http://www.math.uah.edu/stat/interval/Bernoulli.html}
\bibitem{Bib3} \url{https://en.wikipedia.org/wiki/Binomial_proportion_confidence_interval#Wilson_score_interval}
\end{thebibliography}
\end{document}